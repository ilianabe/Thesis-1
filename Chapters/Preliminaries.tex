\chapter*{Preliminaries}
\addcontentsline{toc}{chapter}{Preliminaries}
%
\setcounter{section}{0}
\section{Overview}
%
\section{Hash function}
%
In this section, we will define the syntax and the security model of the cryptographic hash function, as introduced in \cite{Katz:2007:IMC:1206501}. We will slightly change their definition, because we assume no key as input to the hash function. In our case, the only input is a message.
%
\begin{definition}{(Hash function - syntax)} \cite{Katz:2007:IMC:1206501}
A \textbf{hash function} is a probabilistic polynomial-time algorithm $H$
fulfilling the following:
\begin{itemize}
  \item[$\bullet$] There exists a polynomial $l$ such that $H$ is (deterministic) polynomial-time
  algorithm that takes as input any string $x \in { \{ 0,1 \}}^*$, and outputs
  a string $H(x) \in { \{ 0,1 \}}^{l(n)}$.
\end{itemize}
If for every $n$, $H$ is defined only over inputs of length $l'(n)$ and $l'(n) > l(n)$, then
we say that $H$ is a \textbf{fixed-length hash function} with length parameter $l'$.
\end{definition}

Notice that in the fixed-length case we require that $l'$ be greater than $l$. This ensures that the
function is a hash function in the classic sense in that it \textit{compresses} the input. We remark
that in the general case we have no requirement on $l$ because the function takes for input all (finite) binary strings. Thus, by definition, it also compresses.

We will now define security for this model. We begin by defining a game for a hash function $H$, an adversary $\mathcal{A}$ and a security parameter $n$:
\\

\textbf{The collision-finding game ${Hash-coll}_{\mathcal{A},H}(n)$:} \cite{Katz:2007:IMC:1206501}
\begin{enumerate}
  \item The adversary $\mathcal{A}$ outputs a pair $x$ and $x'$. \\
  Formally, $(x,x') \leftarrow \mathcal{A}(s)$.
  \item The output of the experiment is $1$ if and only if $x \neq x'$ and $H(x) = H(x')$. In such a case we say that $\mathcal{A}$ has found a collision.
\end{enumerate}
%
The definition of collision resistance for hash functions states that no efficient adversary can find a collision except with negligible probability.
%
\begin{definition} \cite{Katz:2007:IMC:1206501}
  A hash function $H$ is \textbf{collision resistant} if for all probabilistic polynomial-time adversaries $\mathcal{A}$ there exists a negligible function $\textbf{negl}$ such that

\begin{equation}\label{eqn:collision}
  Pr[{Hash-coll}_{\mathcal{A},H}(n) = 1] \leq \textbf{negl} \: (n)
\end{equation}
\end{definition}
%
%% Memory-hardness
%
%% Password scramblers
%
%% PRF
%
%% Pebbling game
%
%% Information about pebbling algorithms and complexity
