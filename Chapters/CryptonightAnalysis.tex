\chapter*{Cryptonight Analysis}
\addcontentsline{toc}{chapter}{Cryptonight Analysis}
%
%
\setcounter{section}{0}
\section{Description}
In this section we will describe in detail the proposed implementation of the CryptoNight hash function. This function is used in the Monero project in order to achieve \emph{egalitarian mining}. It is easy to understand why we characterize this implementation as \emph{proposed}, since each miner is free to use any implementation he/she can think of, as long as it produces the right result.

In order to prove that a function is \emph{memory-hard} we need to show that no implementation exists, that can produce the same result using less memory without a significant time cost. In other words, using an implementation which needs less memory is not something that can give advantage to some miner because the time factor will make the procedure equally or more "expensive", even if the miner uses parallel computation techniques.

\noindent Reproduced from CryptoNote~\cite{cryptonight}:
\begin{verbatim}
  CryptoNight is a memory-hard hash function. It is designed to be
  inefficiently computable on GPU, FPGA and ASIC architectures.
\end{verbatim}
In the proposed implementation, a scratchpad\footnote{a large area of memory used to store intermediate values during the evaluation of a memory-hard function.} is used (2MB) to ensure that the memory needed fits the size of L3 cache (per core) in modern processors. In practice, the miner should measure mining power and calculate efficiency. In Monero mining, $CPUs$ cores are only efficient if they can use the super fast 2MB cache over and over. Each core needs about 2MB for CryptoNight to stay cached. So a miner should check how much L2 cache or in rare cases also L3 cache the CPU has. Then divide by 2MB this, will be how many cores he/she can run at the same time.

There are several reasons to suspect that CryptoNight could be a memory-hard function. One of the most popular argument was that a megabyte of internal memory is an almost unacceptable size for a modern ASIC pipeline. But, hardware is evolving and eventually there was recently an effort for Monero \emph{ASIC} production.

There were some thoughts like "\emph{How did they did this? Isn't CryptoNight memory-bound?}". Well, one thing is that CryptoNight is \emph{ASIC-resistant}, not \emph{proof}. But, that was not the case. Another issue is that L3 cache supports a lot of extra functionality like being shared across cores, writing back to $RAM$, being behind two other levels of cache, etc. which all makes a lot less efficient (among other issues with the approach). But, again, that was not the case in that particular effort.

L3 latency wasn't the issue. The $ASICs$ just traded latency for bandwidth the same way $GPUs$ do. They're built on stacks of $DRAM$ not lightning fast caches. The costs of cache complexity aren't only latency but also power usage and die space. Raw speed isn't even necessarily the goal for either $CPUs$ or $GPUs$ or $ASICs$ here, it is efficiency.

But, Monero project reacted and announced upgrades bi-annually in order to keep ASIC's at bay. Upgrades are a problem, because upgrades produce bugs and vulnerabilities. Especially, when they are that frequent. On the other hand, upgrades in Monero are minor with no changes to the memory-hard part. From this experience we understand that a formal proof or even a better understanding of the memory-hardness property in practice, is vital in order to protect a cryptocurrency from centralization.

In this chapter we will just show the proposed implementation of CryptoNight without any analysis. We will demonstrate the three stages of the computation and the role for each element. A really quick overview of these stages would be something like this:

\begin{enumerate}
  \item Initialize the scratchpad in a pseudo-random manner.
  \item Read/write operations at pseudo-random addresses. (memory-hard part)
  \item Use all the computations' results to produce the output.
\end{enumerate}

\section{The three Stages}
Enough with the overview of the function and its history. Let's dive into it and see in detail its components as it is described in \cite{cryptonight}.

The input of this algorithm is a block and if the value of the Cryptonight function satisfies the target (\hyperref[eq:target]{equation}~\ref{eq:target}), it is possible that this block is the next block in the blockchain.

\begin{equation}
  \label{eq:target}
  \color{Bittersweet} \mbox{Cryptonight}
  \color{black} (
  \color{RedViolet} \mbox{block}
  \color{black} )\leq
  \color{ForestGreen} \mbox{Target}
  \color{black}
\end{equation}
So, the input of the function is a block of transactions along with the necessary fields, which are specified by the Monero protocol.

\subsection{The first Stage}
The first stage of the algorithm sets the initial value of the scratchpad. In order to prevent several attack schemes, the scratchpad must be initialized with data chosen in a way, which is indistiguishable from the uniform distribution. This is the goal.

We will describe the first Stage in several parts and discuss the role of each part and its contribution regarding the properties of function's output. Let us begin:

\noindent Prepare the tools:
% Prepare the tools
\begin{enumerate}
  \item Hash the input using Keccak~\cite{keccak} ($b=1600$, $c=512$).
  \item Choose the first $32$ bytes of the final state.
  \item Interpret them as an AES-256 key.
  \item Expand them to $10$ round keys.
\end{enumerate}
% Comment on the above
Keccak is the versatile cryptographic function that is most known as SHA-3. The parameter analysis and the description of their part is beyond the scope of this thesis. The reader is refered to their work.

We will consider Keccak a collision-free hash function. The next three steps produce random keys for encryption. We consider these keys random enough in the way that they are interpreted as keys and expanded according to \cite{nla.cat-vn4183631}.

\noindent Create the scratchpad:
\begin{enumerate}
  \setcounter{enumi}{4}
  \item Allocate $2097152$ bytes ($2$MiB).
\end{enumerate}
\noindent The encryption part:
\begin{enumerate}
  \setcounter{enumi}{5}
  \item Split the bytes $64$ to $191$ into $8$ blocks of $16$ bytes each.
  \item \label{step 7} Encrypt the blocks as follows:
    \begin{verbatim}
      for i = 0..9 do:
          block = aes_round(block, round_keys[i])
    \end{verbatim}
\end{enumerate}
\begin{enumerate}
  \setcounter{enumi}{7}
  \item Fill $128$ bytes of the scratchpad with the resulting blocks.
\end{enumerate}
\noindent Repeat:
\begin{enumerate}
  \setcounter{enumi}{8}
  \item With the resulting blocks run \hyperref[step 7]{step}~\ref{step 7} again.
\end{enumerate}
%
Each time 128 bytes are written, they represent the
result of the encryption of the previously written 128 bytes. The
process is repeated until the scratchpad is fully initialized.

\subsection{The second Stage (memory-hardness)}
The second stage of the algorithm uses the initialized scratchpad and two values that are computed from the hashed input of the function. Its goal is to perform computations on the scratchpad values (all of them, with high probability) and produce a final scratchpad structure that can't be computed otherwise or in stages (without huge time complexity). The memory-hardness property is satisfied if and only if there is no other way to compute the final stage of the scratchpad, using memory less than the size of the scratchpad. That is the intuition. In detail:

\noindent \emph{(The preparation part).} The core structure of this stage is a loop. However, before illustrating the computations that take place inside the loop, there are some computations needed for preparation and two technical clarifications.

\begin{enumerate}
  \item \label{memh: step 1} Compute the values of $a$ and $b$.
\end{enumerate}
Elements $a$ and $b$ are the two values which, along with the scratchpad, are given as input to the loop. More specifically, the first $64$ bytes of the hashed input (the Keccak state) are split in two parts ($32$ bytes each part) and $XOR$ed, and the resulting $32$ bytes are used to initialize variables $a$ and $b$, $16$ bytes each.

\noindent \emph{(Clarification 1).} The reader may notice in figure ... that the function uses a 16-byte value as an address in the scratchpad. Actually, the value is interpreted as a little-endian integer. The $21$ low-order bits are used as a byte index. To ensure the 16-byte alignment, the four low-order bits of the index are cleared. This alignment is essential, as the data is read from and written to the scratchpad in 16-byte blocks.

\noindent \emph{(Clarification 2).} The main loop is iterated $524,288 = 2^{19}$ times. Every time, two blocks of the scratchpad are written, so with high probability, the whole scratchpad will be overwritten. In every iteration, along with the two blocks of the scratchpad, values $a'$ and $b'$ are computed, which are used as input to the next iteration.

Now we are ready to describe the inner computations of the loop. We will divide this stage into parts, as it will help us later in the analysis. The number of parts are determined, based on some intermediate values. Here, we note that the intermediate values are not memory requirements, as we can implement the computations with only $32$ bytes of memory (for $a$ and $b$) plus the memory needed for the scratchpad. But during theoretical analysis and understanding of the function's computations, these intermediate values seem natural stops of the train of thought. So, after the \hyperref[memh: step 1]{step}~\ref{memh: step 1}:

\begin{enumerate}
  \setcounter{enumi}{1}
  \item \label{memh: step 2} Interpret the value of $a$ as a scratchpad address.
  \item \label{memh: step 3} Read from this address.
  \item Evaluate the AES function with data from \hyperref[memh: step 3]{step}~\ref{memh: step 3} and key the value of $a$.
\end{enumerate}
Let's call this intermediate value $c$. And let's add a final step to this part:
\begin{enumerate}
  \setcounter{enumi}{4}
  \item $XOR$ $c$ with $b$ and write the result to the address of \hyperref[memh: step 2]{step}~\ref{memh: step 2}.
\end{enumerate}
The value $c$ is passed as $b'$, part of the input of the next iteration. The second part involves another read from the scratchpad:

\begin{enumerate}
  \setcounter{enumi}{5}
  \item \label{memh: step 6} Interpret the value of $c$ as a scratchpad address.
  \item Read from this address. (We will refer to this intermediate value as $d$.)
  \item \label{memh: step 8} Multiply\footnote{The multiplication uses only the first 8 bytes of each argument, which are interpreted as unsigned 64-bit little-endian integers and multiplied together. The result is converted into 16 bytes, and finally the two 8-byte halves of the result are swapped~\cite{cryptonight}.} $c$, $d$ and add the value of $a$ to the result.
  \item Write the result of \hyperref[memh: step 8]{step}~\ref{memh: step 8} to the address of \hyperref[memh: step 6]{step}~\ref{memh: step 6}.
\end{enumerate}
This concludes the second part. The scratchpad is written twice per iteration. The only thing that is left to conclude the second stage is the computation of $a'$, part of the input of the next iteration. This is computed as follows:

\begin{enumerate}
  \setcounter{enumi}{9}
  \item $XOR$ $d$ with the result of \hyperref[memh: step 8]{step}~\ref{memh: step 8} to compute $a'$.
\end{enumerate}



%
%% The above is not done. Check again tomorrow from line 50. I leave now to drink a fucking beer.. Cheers tomorrow's self...
%
%% Here my thought is: Describe as an algorithm (steps) the procedure and maybe along a figure. Just a thought...
%
%% Description (Stage 1,2,3)
%
%% Stage 2 and Memory-hardness
%
%% Details and Graphics representation
