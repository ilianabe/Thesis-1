\chapter*{Cryptocurrency Mining}
\addcontentsline{toc}{chapter}{Cryptocurrency Mining}
%
\setcounter{section}{0}
\section{Overview}
%
\section{Bitcoin}
Bitcoin~\cite{Nakamoto_bitcoin:a} is a decentralized digital currency that enables instant payments to anyone, anywhere in the world. Bitcoin uses peer-to-peer technology to operate with no central authority: transaction management and money issuance are carried out collectively by the network.

The original Bitcoin software by Satoshi Nakamoto was released under the MIT license. Most client software, derived or "from scratch", also use open source licensing.

Bitcoin is the first successful implementation of a distributed crypto-currency, described in part in 1998 by Wei Dai on the cypherpunks mailing list. Building upon the notion that money is any object, or any sort of record, accepted as payment for goods and services and repayment of debts in a given country or socio-economic context, Bitcoin is designed around the idea of using cryptography to control the creation and transfer of money, rather than relying on central authorities.

Bitcoins have all the desirable properties of a money-like good. They are portable, durable, divisible, recognizable, fungible, scarce and difficult to counterfeit.

%
\section{Mining}
Bitcoin mining\footnote{It is misleading to think that there is an analogy between gold mining and bitcoin mining. The fact is that gold miners are rewarded for producing gold, while bitcoin miners are not rewarded for producing bitcoins; they are rewarded for their record-keeping services.} is the processing of transactions in the digital currency system, in which the records of current Bitcoin transactions, \textit{blocks}, are added to the record of past transactions, the \textit{blockchain}. Miners keep the blockchain consistent, complete, and unalterable by repeatedly grouping newly broadcast transactions into a block, which is then broadcast to the network and verified by recipient nodes.~\cite{economist} Each block contains a SHA-256 cryptographic hash of the previous block~\cite{economist}, thus linking it to the previous block and giving the blockchain its name.

To be accepted by the rest of the network, a new block must contain a proof-of-work ($PoW$). The system used is based on Adam Back's 1997 anti-spam scheme, Hashcash~\cite{Nakamoto_bitcoin:a}.
The $PoW$ requires miners to find a number called a nonce, such that when the block content is hashed along with the nonce, the result is numerically smaller than the network's difficulty target~\cite{Nakamoto_bitcoin:a}. This proof is easy for any node in the network to verify, but extremely time-consuming to generate, as for a secure cryptographic hash, miners must try many different nonce values before meeting the difficulty target.

The primary purpose of mining is to set the history of transactions in a way that is computationally impractical to modify by any one entity. By downloading and verifying the blockchain, bitcoin nodes are able to reach consensus about the ordering of events in bitcoin.~\cite{wiki}

Every 2,016 blocks (approximately 14 days at roughly 10 min per block), the difficulty target is adjusted based on the network's recent performance, with the aim of keeping the average time between new blocks at ten minutes. In this way the system automatically adapts to the total amount of mining power on the network. Between 1 March 2014 and 1 March 2015, the average number of nonces miners had to try before creating a new block increased from 16.4 quintillion to 200.5 quintillion.\cite{difficulty_history}

The proof-of-work system, alongside the chaining of blocks, makes modifications of the blockchain extremely hard, as an attacker must modify all subsequent blocks in order for the modifications of one block to be accepted. As new blocks are mined all the time, the difficulty of modifying a block increases as time passes and the number of subsequent blocks (also called confirmations of the given block) increases.~\cite{economist}

Mining is also the mechanism used to introduce Bitcoins into the system: Miners are paid any transaction fees as well as a "subsidy" of newly created coins. This both serves the purpose of disseminating new coins in a decentralized manner as well as motivating people to provide security for the system.~\cite{wiki}

Originally, Bitcoin mining was conducted on the \textit{CPUs} of individual computers, with more cores and greater speed resulting in more profitability. After that, the system became dominated by multi-graphics card systems, then field-programmable gate arrays (\textit{FPGAs}) and finally application-specific integrated circuits (\textit{ASICs}), in the attempt to find more hashes per hour with less electrical power usage.

Due to this constant escalation, it has become hard for prospective new miners to start. This adjustable difficulty is an intentional mechanism created to prevent inflation. To get around that problem, individuals often work in mining pools.

Bitcoin generally started with individuals and small organizations mining. At that time, start-up could be enabled by a single high-end gaming system. Now, however, larger mining organizations might spend tens of thousands on one high-performance, specialized application-specific integrated circuit.

That creates a problem. In a system, which from its creation, it is supposed to distribute power among users, there has been a great power concentration in the hands of big companies, like Bitfury or 21, that develop ASICs to mine bitcoin. Because of the extreme cost of ASICs and extreme hashrate, someone who uses a multi-graphics card system or a $CPU$ is out of competition. As a result of this, independent miners have largely dried up.

%
\section{Egalitarian Mining}
Let's consider several contexts where an adversary has an upper hand over the defender by using special hardware in an attack. These include password processing, hard-drive protection, cryptocurrency mining, resourse sharing, code obfuscation, etc. Memory-hard computing is a generic paradigm, which can protect the defender against attacks in the aforementioned contexts. Every task is amalgamated with a certain procedure requiring intensive access to $RAM$ both in terms of size and bandwidth, so that transferring the computation to $GPU$, $FPGA$, and even $ASIC$ brings little or no cost reduction. Cryptographic schemes that run in this framework become \emph{egalitarian} in the sense that both users and attackers are equal in the price-performance ratio conditions. When the cryptographic scheme is a hash fuction used for cryptocurrency mining, then we refer to this notion as \emph{egalitarian mining}.

But let's step back a little and think about the need for such a notion. Do we actually need it? Is egalitarian mining a way to destroy competition? Is it unfair? Shouldn't a miner be rewarded for the extra money he invested?

Many questions like the above have been asked and usually the answer is not descriptive enough of what really memory-hardness introduces to the world. We will try here to demonstrate exactly what it means for a coin to offer egalitarian mining.

\emph{Egalitarian mining does not destroy competition.} The miner who invests more in hardware, is rewarded more. Each individual miner is rewarded according to the computation power he offers to the coin. The real difference is that it is really easy for people to start mining with a single high-end gaming system. Hobbists, who want to support the community are welcome to mine. In bitcoin system, this option is not available. In order to support the coin by mining, you have to invest a lot of money on ASICs to be competitive. That means, that in general, the mining for support or for fun, is dead.

This is hurtful for a system, which by design is supposed to bring decentralization in the financial market. That is because without hobbists we are actually left with big companies handling almost all of the mining. Companies will comply with regulations that the government of each country enforces and cannot be expected to react and inspire movements. Since countries can and they have, historically, collaborated against threats, a union of countries who can enforce regulations to companies, that control more than 51\% of the hashing power, can bring a coin to its knees, if seen as a threat. That scenario, does not fit in most definitions of security.

One of the reasons that coins have a bootstrapping period is because they need a big society to distribute mining, in order to guarantee security. When the total hashing power is a few high-end gaming systems, aquaring 51\% of the hashing power is unfeasible. As the support expands, the security is satisfied for all practical purposes. But when mining is dominated by companies, then a totally trustless system gaves birth to a trusted party. That's against the motivation of the inception of a cryptocoin and it raises questions like "\emph{Why should I trust the mining companies and support this cryptocoin? Do I trust my bank more? After all, my bank is just another company...}"

To sum up, for reasons of security and decentralization, it is healthy for some coin's mining power to be distributed among users. Memory-hardness sustains the competition, but it makes it less harsh and keeps the door open for hobbists to support the coin. It is extremely difficult for the corporate mining to aquire tremendous power for two reasons, and that is needed in a trustless system that wants to stay trustless. The reasons are: a) it is not that lucrative for companies. If someone makes a big investment, he will get big rewards, but not insanely huge rewards leaving every hobbist out of the mining society. b) Even if a lot of companies decide to participate, it is extremely difficult for them to aquire a combined 51\% of the total hash power without destroying the "support mining" or "fun mining".

Memory-hardness defends a system against this prospect and thus strengthens the notion of coin's security. Furthermore, no matter how much someone trusts the corporations involved, security is a binary state. A system cannot be secure against attackers and insecure against other parties. It is either secure or insecure. And if the prospect of an attack exists, then security collapses.
%
\section{Monero}
Monero (XMR) is a decentralised open-source cryptocurrency forked from Bytecoin in April 2014. The coin's fundamental feature is privacy - it aims to be a digital medium of exchange with untraceable payments, unlinkable transactions and resistance to blockchain analysis. The exact person behind a Monero transaction is not known; this results in considerable increase of privacy compared to Bitcoin and its forks.~\cite{monerodef}

Monero is actively encouraged to those seeking financial privacy, since payments and account balances remain entirely hidden, which is not the standard for most cryptocurrencies.\\

\noindent Monero is:~\cite{monero}
\begin{description}
  \item [Untraceable] Monero uses a digital signature scheme called "ring signatures", which shuffles users' public keys in order to eliminate the possibility to identify a particular user.
  \item [Unlinkable] Monero employs a specific protocol which generates multiple unique one-time addresses that can only be linked by the payment receiver and are unfeasable to be revealed through blockhain analysis.
  \item [Secure] Monero is cryptographically secured. Moreover, the design of the algorithm used, consists in tremendous computational and electric capibilities that an adversary would need to even try to steal funds.
  \item [Private] Privacy is basically provided with the idea of anonymous transactions without any obligations to cooperate with third parties.
  \item [Analysis Resistant] Monero's blockhain analysis resistantce results from unlinkability, which was achieved by using a modified version of the Diffie-Hellman exchange protocol that generates multiple one-time public addresses that can only be simply gathered by the message receiver, but hardly analyzed by confused foreigners inside the block explorer.
\end{description}
%
Monero uses a Proof of Work mechanism to issue new coins and
incentivize miners to secure the network and validate transactions.
One key part, for Monero to offer the above, is a proof-of-work algorithm called CryptoNight, developed by the CryptoNote
project~\cite{citeulike:14139412}. On top of typical security attributes, this algorithm is also suspected to be memory-hard. The aim of this work is to study the memory-hardness property of this algorithm.
%
%% Cryptonight
